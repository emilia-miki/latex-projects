\documentclass[12 pt, a4paper]{article}

\usepackage{preamble}

\begin{document}

\large
\onehalfspacing

\thispagestyle{empty}

\begin{center}

    \bfseries

    НАЦІОНАЛЬНИЙ ТЕХНІЧНИЙ УНІВЕРСИТЕТ УКРАЇНИ

    "КИЇВСЬКИЙ ПОЛІТЕХНІЧНИЙ ІНСТИТУТ

    імені  ІГОРЯ  СІКОРСЬКОГО"

    Інститут  прикладного системного аналізу

    Кафедра системного проектування

    \mdseries

    \vfill

    Звіт

    про виконання лабораторної роботи №2

    з дисципліни "Комп‘ютерні мережі"

    \vfill

    \begin{flushleft}

        Виконав:

        студент III курсу, групи ДА-01

        Дячина Микита

    \end{flushleft}

    \vfill

    Київ - 2023

\end{center}

\newpage

\section*{Мета роботи}

\begin{itemize}
    \item ознайомлення із засобами перехоплення, збереження і аналізу мережевого трафіку за допомогою відомих аналізаторів мережевого трафіку як Wireshark;

    \item проведення аналізу мережевого сеансу транспортного протоколу TCP.
\end{itemize}

{\bfseries Варіант 8.}

\section*{Хід роботи}

Спочатку використаємо ipconfig для отримання інформації про інтерфейс en0
(wi-fi, що використовується для доступу до інтернету):

\begin{minted}{text}
~>ipconfig getiflist
en3 en4 bridge0 en0
~>ipconfig getsummary en0
<dictionary> {
  BSSID : ac:84:c6:84:0e:10
  IPv4 : <array> {
    0 : <dictionary> {
      Addresses : <array> {
        0 : 192.168.1.101
      }
      ChildServiceID : LINKLOCAL-en0
      ConfigMethod : DHCP
      DHCP : <dictionary> {
        LeaseIsInfinite : TRUE
        LeaseStartTime : 04/18/2023 12:18:46
        Packet : op = BOOTREPLY
htype = 1
flags = 0
hlen = 6
hops = 0
xid = 0x706b3e93
secs = 0
ciaddr = 0.0.0.0

      IsPublished : TRUE
      Router : 192.168.1.1
      RouterARPVerified : TRUE
      ServiceID : EA7DD607-2890-4A6C-A307-EBBC2FD17889
      SubnetMasks : <array> {
        0 : 255.255.255.0
      }
    }
    1 : <dictionary> {
      ConfigMethod : LinkLocal
      IsPublished : TRUE
      ParentServiceID : EA7DD607-2890-4A6C-A307-EBBC2FD17889
      ServiceID : LINKLOCAL-en0
    }
  }
  IPv6 : <array> {
    0 : <dictionary> {
      ConfigMethod : Automatic
      DHCPv6 : <dictionary> {
        Mode : None
        State : Inactive
      }
      IsPublished : FALSE
      LastFailureStatus : network changed
      RTADV : <dictionary> {
        State : Solicit
      }
      ServiceID : EA7DD607-2890-4A6C-A307-EBBC2FD17889
    }
  }
  InterfaceType : WiFi
  LinkStatusActive : TRUE
  NetworkID : 6E18AB8F-4B58-4428-9CBF-35B4AB5DB612
  SSID : TP-Link_0E10
  Security : WPA2_PSK
}
\end{minted}

Звідси видно, що IP-адреса комп‘ютера в локальній мережі –  192.168.1.101, і MAC-адреса мережевого адаптера – 1c:91:80:c9:fb:2f.

Тепер проведемо коротке захоплення пакетів і переглянемо отриману статистику:



\section*{Висновок}

\end{document}

